\textbf{- Reference:} 
\href{https://www.zama.ai/post/tfhe-deep-dive-part-3}{TFHE Deep Dive - Part III - Key switching and leveled multiplications}~\cite{tfhe-3}

$ $

Key switching is a process to change a GLWE ciphertext's secret key from $S$ to a new key $S'$ without decrypting the ciphertext. This is equivalent to converting a ciphertext $\textsf{GLWE}_{S,\sigma}(\Delta M)$ into a new ciphertext $\textsf{GLWE}_{S',\sigma}(\Delta M)$. 

$ $

\noindent Remember that $\textsf{GLWE}_{S,\sigma}(\Delta M) = (A_0, A_1, \gap{$\cdots$} , A_{k-1}, B) \in \mathcal{R}_{\langle n, q \rangle }^{k + 1}$

, where $B = \sum\limits_{i=0}^{k-1}{(A_i \cdot S_i)} + \Delta \cdot M + E$

, and the secret key $S$ is a list of $k$ polynomials: $S = (S_0, S_1, \gap{$\cdots$} S_{k-1})$

$ $

\noindent Also, remember that GLev (\autoref{sec:glev}) is defined as follows: 

$\textsf{GLev}_{S, \sigma}^{\beta, l}(M) = \Bigl \{\text{GLWE}_{S, \sigma}\Bigl (\dfrac{q}{\beta^i} \cdot M \Bigr ) \Bigr \}_{i=1}^{l} \in \mathcal{R}_{\langle n, q \rangle }^{(k+1) \cdot l}$

$ $

\noindent Now, let's denote each of the $k$ key-switching keys as follows:

$\mathit{KSK}_i = \textsf{GLev}_{S', \sigma}^{\beta, l}(S_i)$ \\
$ = \left ( \textsf{GLWE}_{S', \sigma} \left ( 
 \dfrac{q}{\beta^1}S_i \right ), \textsf{GLWE}_{S', \sigma} \left ( 
 \dfrac{q}{\beta^2}S_i \right ), \gap{$\cdots$}, \textsf{GLWE}_{S', \sigma} \left ( 
 \dfrac{q}{\beta^l}S_i \right ) \right ) \in \mathcal{R}_{\langle n, q \rangle }^{(k + 1) \cdot l}$

$ $

\noindent, which is a list of GLWE encryptions of the secret key $S$ by $S'$. Then, the GLWE ciphertext's key can be switched from $S \rightarrow S'$ as follows: 


\begin{tcolorbox}[title={\textbf{\tboxlabel{\ref*{sec:glwe-key-switching}} GLWE Key Switching}}]
$\textsf{GLWE}_{S',\sigma}(\Delta M) = (\overbrace{0, 0, \gap{$\cdots$} 0}^k, B) - \sum \limits_{i=0}^{k-1} A_i \cdot S_i$

\text{ } \text{ } $ = (0, 0, \gap{$\cdots$} 0, B) - \sum \limits_{i=0}^{k-1} \langle \textsf{Decomp}^{\beta, l}(A_i), \mathit{KSK}_i \rangle$
\end{tcolorbox}



\begin{myproof}

\begin{enumerate}
\item Given the principle of polynomial decomposition (\autoref{subsec:poly-decomp}) and the polynomial $A_i \in \mathbb{Z}_q[x] / (x^n + 1)$, its decomposition is as follows: 

$ $

$\textsf{Decomp}^{\beta, l}(A_i) = (A_{\langle i,1 \rangle}, A_{\langle i,2 \rangle}, \gap{$\cdots$} A_{\langle i,l \rangle})$, where

$ $

$A_i = A_{\langle i, 1 \rangle} \dfrac{q}{\beta^1} + A _{\langle i, 2 \rangle}\dfrac{q}{\beta^2} + \cdots + A_{\langle i, l \rangle} \dfrac{q}{\beta^l}  $

$ $

\item $\langle \textsf{Decomp}^{\beta, l}(A_i), \mathit{KSK}_i \rangle $ \\
$ =  A_{\langle i,1 \rangle} \cdot \textsf{GLWE}_{S', \sigma} \left ( 
 \dfrac{q}{\beta^1}S_i \right ) + A_{\langle i,2 \rangle} \cdot \textsf{GLWE}_{S', \sigma} \left ( 
 \dfrac{q}{\beta^2}S_i \right ) + \gap{$\cdots$} + A_{\langle i,l \rangle} \cdot \textsf{GLWE}_{S', \sigma} \left ( 
 \dfrac{q}{\beta^l}S_i \right ) $ 
\textcolor{red}{ \# where each \textsf{RLWE} ciphertext is an encryption of $S_i\dfrac{q}{\beta}, S_i\dfrac{q}{\beta^2}, \cdots, S_i\dfrac{q}{\beta^l}$ as plaintext with the plaintext scaling factor 1}

$ $
 
$= \textsf{GLWE}_{S', \sigma} \left ( 
 \left ( A_{\langle i,1 \rangle} \dfrac{q}{\beta^1} + A_{\langle i,2 \rangle} \dfrac{q}{\beta^2} + \gap{$\cdots$} A_{\langle i,l \rangle} \dfrac{q}{\beta^l} \right ) \cdot S_i \right )$ 

 
$= \textsf{GLWE}_{S', \sigma} \left ( 
 \left ( A_i \right ) \cdot S_i \right ) = \textsf{GLWE}_{S', \sigma} ( 
 A_i  S_i ) $ 

$ $
 
 \item $\sum \limits_{i=0}^{k-1} \langle \textsf{Decomp}^{\beta, l}(A_i), \mathit{KSK}_i \rangle = \sum \limits_{i=0}^{k-1} (\textsf{GLWE}_{S', \sigma} ( 
 A_i  S_i )) $ 
 
 $= \textsf{GLWE}_{S', \sigma} \left ( \sum \limits_{i=0}^{k-1}  
 A_i  S_i  \right ) $

$ $
 
 \item $B$ is equivalent to a trivial GLWE encryption with $S'$, where its every $A_1, A_2, \cdots$ is 0. That is, $\textsf{GLWE}_{S', \sigma}(B) = (\overbrace{0, 0, \gap{$\cdots$}}^k, B)$. Note that $\textsf{GLWE}_{S', \sigma}(B)$ can be created without the knowledge of $S'$, because its all $A_iS_i$ terms are $0$. 

$ $
 
 \item $\textsf{GLWE}_{S', \sigma}(B) - \textsf{GLWE}_{S', \sigma} \left( \sum \limits_{i=0}^{k-1} (A_i  S_i )\right) = \textsf{GLWE}_{S', \sigma}(B - \sum \limits_{i=0}^{k-1} (A_i  S_i )) = \textsf{GLWE}_{S', \sigma}(\Delta M + E)$

$ $

\item To expand the above relation, 

$\textsf{GLWE}_{S', \sigma}(B) - \textsf{GLWE}_{S', \sigma} \left(\sum \limits_{i=0}^{k-1} (A_i  S_i)\right)$

$ = (\overbrace{0, 0, \gap{$\cdots$}}^k, B) - (\overbrace{A'_0, A'_1, \gap{$\cdots$}, A'_k}^k, B')  $ \text { } \textcolor{red}{ \# where $B' = \sum\limits_{i=0}^{k-1} A'_iS'_i + \sum\limits_{i=0}^{k-1} A_iS_i + E'$}

$(\overbrace{-A'_0, -A'_1, \gap{$\cdots$}, -A'_k}^k, \text{ } B - B')$

$ $

The decryption of the above ciphertext gives us:

$ \textsf{GLWE}_{S', \sigma}^{-1} \bm{\Big(} \text{ } (\overbrace{-A'_0, -A'_1, \gap{$\cdots$}, -A'_k}^k, \text{ } B - B') \text{ } \bm{\Big)}$

$ = B -  B' - \sum\limits_{i=0}^{k-1} -A'_iS_i'$

$= B - \sum\limits_{i=0}^{k-1} A'_iS'_i - \sum\limits_{i=0}^{k-1} A_iS_i - E' + \sum\limits_{i=0}^{k-1} A'_iS_i'$

$= B - \sum\limits_{i=0}^{k-1} A_iS_i - E'$

$= \Delta M + E - E' \approx \Delta M + E$

Strictly speaking, $B - \sum\limits_{i=0}^{k-1} A_iS_i = \Delta M + E + Kq$ where $K$ is a polynomial to represent the wrap-around coefficient values as multiples of $q$. However, since all the above computations are done in modulo $q$, the $Kq$ term gets eliminated. 

$ $

\item Thus, $ (0, 0, \gap{$\cdots$} , B) - \sum \limits_{i=0}^{k-1} \langle \textsf{Decomp}^{\beta, l}(A_i), \mathit{KSK}_i \rangle \approx \textsf{GLWE}_{S',\sigma}(\Delta M) $ 

$\textsf{GLWE}_{S',\sigma}(\Delta M)$ is an encryption of plaintext $\Delta M$ with the plaintext scaling factor 1. However, we can theoretically see this as an encryption of plaintext $M$ with the plaintext scaling factor $\Delta$. This way, we can recover the ciphertext's original scaling factor $\Delta$ without any additional computation. 
\end{enumerate}
\end{myproof}
