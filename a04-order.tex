\textbf{- Reference:} 
\href{https://e.math.cornell.edu/people/belk/numbertheory/CyclotomicPolynomials.pdf}{Fields and Cyclotomic Polynomials}~\cite{cyclotomic-polynomial}

\subsection{Definitions}
\label{subsec:order-def}

\begin{tcolorbox}[title={\textbf{\tboxdef{\ref*{subsec:order-def}} Order Definition}}]
$\bm{\textsf{ord}_\mathbb{F}(a)}$: For $a \in \mathbb{F}$ (a finite field, \autoref{subsec:field-def}), $a$'s order is the smallest positive integer $k$ such that $a^k = 1$. 

$ $


\end{tcolorbox}

Note that the multiplicative group generated by $a$ as a generator excludes $\{0\}$, the identity, because $0^k = 0$ for all $k$ values.

\subsection{Theorems}
\label{subsec:order-theorem}



\begin{tcolorbox}[title={\textbf{\tboxtheorem{\ref*{subsec:order-theorem}.1} Order Property (I)}}]
For $a \in \mathbb{F}$, and $n \geq 1$, $a^n = 1$ if and only if $\textsf{ord}_\mathbb{F}(a) \text{ } | \text{ } n$ 

(i.e., $\textsf{ord}_\mathbb{F}(a)$ divides $n$).
\end{tcolorbox}

\begin{myproof}
    \begin{enumerate}
    \item \textit{Forward Proof:} If $\textsf{ord}_\mathbb{F}(a) \text{ } | \text{ } n$, then for $\textsf{ord}_\mathbb{F}(a) = k$ where $k$ is $a$'s order, and $n = lk$ for some integer $l$. 
    
    Then, $a^n = a^{lk} = (a^k)^l = 1^l = 1$.
    \item \textit{Backward Proof:} If $a^n = 1$, then for $\textsf{ord}_\mathbb{F}(a) = k$, $n \geq k$, because by definition of order, $k$ is the smallest value that satisfies $a^k = 1$. For any $n > k$, $n$ has to be a multiple of $k$, because in order to satisfy $a^n = a^k \cdot a^{n - k} = 1 \cdot a^{n - k} = 1$, the next smallest possible value for $n$ is $2k$ (as $a^k$ is the smallest value that is equal to 1). By induction, the possible values of $n$ are $n = k, 2k, 3k, \cdots$.
    \end{enumerate}
\end{myproof}


\begin{tcolorbox}[title={\textbf{\tboxtheorem{\ref*{subsec:order-theorem}.2} Order Property (II)}}]
If $\textsf{ord}_\mathbb{F}(a) = k$, then for any $n \geq 1$, $\textsf{ord}_\mathbb{F}(a^n) = \dfrac{k} {\text{gcd}(k, n)}$.
\end{tcolorbox}
\begin{myproof}
    \begin{enumerate}
    \item $a^k, a^{2k}, a^{3k}, ... \text{ } = 1$. 
    \item Given $\textsf{ord}_\mathbb{F}(a^n) = m$, $(a^n)^m, (a^n)^{2m}, (a^n)^{3m}, ... \text{ } = 1$ 
    \item Note that by definition of order, $x=k$ is the smallest value that satisfies $a^x$ = 1. Thus, given $\textsf{ord}_\mathbb{F}(a^n) = m$, then $m$ is the smallest integer that makes $(a^n)^m = 1$. Note that $(a^n)^m$ should be a multiple of $a^k$, which means $mn$ should a multiple of $k$. The smallest possible integer $m$ that makes $mn$ a multiple of $k$ is $m = \dfrac{k}{\text{gcd}(k, n)}$. 
    \end{enumerate}
\end{myproof}

\begin{tcolorbox}[title={\textbf{\tboxtheorem{\ref*{subsec:order-theorem}.3} Order Property (III)}}]
Given $k \text{ } | \text{ } n$, $\textsf{ord}_\mathbb{F}(a) = kn$ if and only if $\textsf{ord}_\mathbb{F}(a^k) = n$.
\end{tcolorbox}
\begin{myproof}
\begin{enumerate}
    \item \textit{Forward Proof:} Given $\textsf{ord}_\mathbb{F}(a) = kn$, and given Theorem~\ref*{subsec:order-theorem}.2, $\textsf{ord}_\mathbb{F}(a^k) = \dfrac{nk}{\text{gcd}(k, nk)} = \dfrac{nk}{k} = n$.
    \item \textit{Backward Proof:} Given $\textsf{ord}_\mathbb{F}(a^k) = n$, we should prove that $\textsf{ord}_\mathbb{F}(a) = nk$. Let $\textsf{ord}_\mathbb{F}(a) = m$. Then, by Theorem\ref*{subsec:order-theorem}.2, $\textsf{ord}_\mathbb{F}(a^k) = \dfrac{m}{\text{gcd}(m, k)}$, which is $n$. In other words, $m = n \cdot \text{gcd}(m, k)$. But as $k \text{ } | \text{ } n$, it is also true that $k \text{ } | \text{ } m$. Then, $\text{gcd}(m, k) = k$. Therefore, $m = n \cdot \text{gcd}(m, k) = nk$.
\end{enumerate}
\end{myproof}

\begin{tcolorbox}[title={\textbf{\tboxtheorem{\ref*{subsec:order-theorem}.4} Fermat's Little Theorem}}]
Given $|\mathbb{F}| = p$ (a prime) and $a \in \mathbb{F}$, $a^p = a$.
\end{tcolorbox}
\begin{myproof}
    \begin{enumerate}
    \item $\{a^1, a^2, ... \text{ } a^p\}$ forms a multiplicative subgroup $H$ of the group $G = \mathbb{F^{\times}}$ (i.e., $\mathbb{F}$ without $\{0\}$). 
    \item Lagrange's Group Theory states that for any subgroup $H$ in a group $G$, $|H|$ divides $|G|$. This means that given $\textsf{ord}_G(a) = k$ (where $a^k = 1$), $k$ divides $|G|$ (where $|G| = p - 1$). 
    \item Therefore, given $kl = p - 1$ for some integer $l$, $a^{p-1} = (a^k)^l = 1^l = 1$. Thus, $a^p = a$.
    \end{enumerate}
\end{myproof}
