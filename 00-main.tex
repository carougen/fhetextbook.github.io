\PassOptionsToPackage{colorlinks=true,linkcolor=blue,urlcolor=blue}{hyperref}



\documentclass[11pt]{article}

%\usepackage[top=1in, bottom=1in, left=1cm, right=1cm]{geometry}
%\usepackage{lipsum}
%\usepackage{background}
%\backgroundsetup{
%  position=current page.east,
%  angle=-90,
%  nodeanchor=east,
%  vshift=-5mm,
%  opacity=1,
%  scale=3,
%  contents=Confidential
%}

\usepackage{amsmath,amssymb,amsthm,bm}   % TeX4ht relies on these
\usepackage{mathtools}            % optional but harmless
\usepackage{rotating}

\usepackage[margin=1cm, paperwidth=8.5in, paperheight=11in]{geometry}
\usepackage[
  hypertexnames=false
]{hyperref}
\usepackage{kotex}
\usepackage{etex}
\usepackage{fullpage}
\usepackage{mathtools}
\usepackage{amsfonts}
\usepackage{graphicx}
\usepackage{amsthm}
\usepackage[utf8]{inputenc}
\usepackage{afterpage}
\usepackage{adjustbox}
\usepackage{placeins}
\usepackage{fixltx2e}
\usepackage{supertabular} 
\usepackage{titlesec}
\usepackage{array}
\usepackage{tabularx}
\usepackage{color}
\usepackage{algorithm}
\usepackage{algpseudocode}
\usepackage{subcaption}
\usepackage{hyperref}
\usepackage{pgfplots}
\usepackage{relsize}
\let\labelindent\relax
\usepackage{enumitem}
\usepackage{fancyhdr}
\usepackage{alltt}
\usepackage{soul}
\usepackage{fancyvrb}
\usepackage{xcolor}
%\usepackage[english,greek]{babel}
%\usepackage{ucs} 
%\usepackage[utf8x]{inputenc}
%\usepackage[usenames,dvipsnames]{xcolor}
%\usepackage{tikz}
%\usepackage{tkz-tab}
%\usepackage{caption}
%\usepackage{latexsym}
%\usepackage{amssymb}
\usepackage[margin=0.1cm]{subcaption}
\usepackage{multicol}
\usepackage{refcount}

\usepackage{breakurl} % Break URL
\usepackage{multirow}
\usepackage[most]{tcolorbox}
%\usepackage[breakable,skins]{tcolorbox}

\tcbset{breakable}
\usepackage{esvect}
\usepackage{hyperref}
\usepackage{mathdots}
\usepackage{pifont} 
\usepackage{booktabs}
\usepackage{arydshln}
\usepackage{xcolor}
\def\UrlBreaks{\do\/\do-}
\usepackage{mathtools}  
\usepackage{array}
\usepackage{hyperref}
\usepackage{hyperref}
\usepackage{mathtools, nccmath}
%\usepackage[style=numeric,sorting=none]{biblatex}

\usepackage{titling}
\renewcommand\maketitlehooka{\null\mbox{}\vfill}
\renewcommand\maketitlehookd{\vfill\null}

\DeclarePairedDelimiter{\nint}\lfloor\rceil

\DeclarePairedDelimiter{\abs}\lvert\rvert
\usepackage{tocloft}
\usepackage{pdflscape}
\usepackage[all=normal, floats, bibnotes, wordspacing, charwidths, indent]{savetrees}
\makeatletter%
\captionsetup{belowskip=0pt}

\usepackage{enumitem}
\setlist{nolistsep}
%\setlist[itemize]{leftmargin=0.1}
\setlist[itemize]{leftmargin=*}
\setlist[enumerate]{leftmargin=*}

% align figures to the top margin
\makeatletter
\setlength{\@fptop}{0pt}

\newcommand\bunderline{}% check for being undefined
\DeclareRobustCommand\bunderline[1]{\mathord{\mathpalette\b@underline{#1}}}
\newcommand{\b@underline}[2]{%
  \sbox\z@{$\m@th#1\underline{#2}$}%
  \raisebox{-\dp\z@}{\scalebox{.5}[.25]{$\m@th#1[$}}%
  \copy\z@
  \raisebox{-\dp\z@}{\scalebox{.5}[.25]{$\m@th#1]$}}%
}


\makeatother

%\renewcommand{\figureautorefname}{foo}
%\renewcommand{\tableautorefname}{bar}

%\usepackage[all=normal, floats, bibnotes, wordspacing, charwidths, indent, lists]

%\captionsetup{belowskip=0pt}

\usepackage{graphicx}
\def\rotatecharone#1{\rotatebox[origin=c]{180}{#1}}

\def\rotatechartwo#1{\reflectbox{#1}}
\usepackage{hyperref}
\skip\footins 0.5\baselineskip %8.1pt plus 4pt minus 2pt
\floatsep 0.5\baselineskip
\textfloatsep 0.5\baselineskip
\intextsep 0.5\baselineskip 
\dbltextfloatsep 0.5\baselineskip  
\dblfloatsep 0.5\baselineskip 

%\raggedbottom

\belowcaptionskip 0pt

\everypar{\looseness=-1 }

\setlength{\abovecaptionskip}{0pt} % 

\newenvironment{myproof}[1][\proofname]{%
  \begin{proof}[#1]$ $\par\nobreak\ignorespaces
}{%
  \end{proof}
}
\makeatletter
\renewcommand{\sectionautorefname}{\S\@gobble}
\renewcommand{\subsectionautorefname}{\S\@gobble}
\renewcommand{\subsubsectionautorefname}{\S\@gobble}
\renewcommand{\appendixautorefname}{\S\@gobble}
\renewcommand{\appendixautorefname}{\S\@gobble}

\newcommand\mymathcolor[2]{{\color{#1}#2}}

\makeatother
\newcommand{\ignore}[1]{}

% Add additional comma in the math mode
%\DeclareMathSymbol{,}{\mathpunct}{letters}{"2C}
%\newcommand{\comma}{\mathpunct{\mkern\thinmuskip}}
%\mathcode`,=\string"8000
%{\catcode`,=\active \gdef,{\comma}}


\addtolength{\cftsecnumwidth}{15pt}
\addtolength{\cftsubsecnumwidth}{10pt}
\addtolength{\cftsubsubsecnumwidth}{10pt}

\makeatletter%
\usepackage{stackengine,xcolor}
\input pdf-trans
\newbox\qbox
\def\usecolor#1{\csname\string\color@#1\endcsname\space}
\newcommand\bordercolor[1]{\colsplit{1}{#1}}
\newcommand\fillcolor[1]{\colsplit{0}{#1}}
\newcommand\colsplit[2]{\colorlet{tmpcolor}{#2}\edef\tmp{\usecolor{tmpcolor}}%  
  \def\tmpB{}\expandafter\colsplithelp\tmp\relax%
  \ifnum0=#1\relax\edef\fillcol{\tmpB}\else\edef\bordercol{\tmpC}\fi}
\def\colsplithelp#1#2 #3\relax{%
  \edef\tmpB{\tmpB#1#2 }%
  \ifnum `#1>`9\relax\def\tmpC{#3}\else\colsplithelp#3\relax\fi
}
\newcommand\outline[1]{\leavevmode%
  \def\maltext{#1}%
  \setbox\qbox=\hbox{\maltext}%
  \boxgs{Q q 2 Tr \thickness\space w \fillcol\space \bordercol\space}{}%
  \copy\qbox%
}
\newcommand\mathcalbb[2][1]{%
  \stackengine{0pt}{\def\thickness{.55}\outline{$\mathcal{#2}$}}{\kern.1pt\outline{$\mathcal{#2}$}}{O}{l}{F}{F}{L}}
\bordercolor{black}
\fillcolor{white}
\def\thickness{.1}% TO CHANGE THICKNESS OF SHADOW
\usepackage{lmodern}
\usepackage{lipsum}


\makeatletter
\DeclareRobustCommand{\ccong}{\mathrel{\mathpalette\@verequiv\sim}}
\newcommand{\@verequiv}[2]{%
  \lower.5\p@\vbox{
    \lineskiplimit\maxdimen
    \lineskip-.5\p@
    \ialign{%
      $\m@th#1\hfil##\hfil$\crcr
      #2\crcr
      \equiv\crcr
    }%
  }%
}
\renewcommand*\env@matrix[1][c]{\hskip -\arraycolsep
  \let\@ifnextchar\new@ifnextchar
  \array{*\c@MaxMatrixCols #1}}


\newcommand{\hathat}[1]{% 
\begingroup%
  \let\macc@kerna\z@%
  \let\macc@kernb\z@%
  \let\macc@nucleus\@empty%
  \hat{\raisebox{.2ex}{\vphantom{\ensuremath{#1}}}\smash{\hat{#1}}}%
\endgroup%
}

\newcommand{\revise}[1]{{{\textcolor{black}{#1}}}}

\makeatother

% \title{**\vspace*{\fill}**The Beginner's Textbook for Fully Homomorphic Encryption}

% \date{January 1, 2025}

% \author{Ronny Ko**\vspace*{\fill}**}


\begin{document}
\title{\Huge{\textbf{The Beginner's Textbook}}\\ \Huge{\textbf{for Fully Homomorphic Encryption}}}
\author{\textbf{}\\{LG Electronics Inc.}}%\\\texttt{\small{hajoon.ko@lge.com}}}
\date{January 1, 2025}



\begin{titlingpage}
\maketitle
\end{titlingpage}


\clearpage

\section*{Preface}
Fully Homomorphic Encryption (FHE) is a cryptographic scheme that enables computations to be performed directly on encrypted data, as if the data were in plaintext. After all computations are performed on the encrypted data, it can be decrypted to reveal the result. The decrypted value matches the result that would have been obtained if the same computations were applied to the plaintext data.

FHE supports basic operations such as addition and multiplication on encrypted numbers. Using these fundamental operations, more complex computations can be constructed, including subtraction, division, logic gates (e.g., AND, OR, XOR, NAND, MUX), and even advanced mathematical functions such as ReLU, sigmoid, and trigonometric functions (e.g., sin, cos). These functions can be implemented either as exact formulas or as approximations, depending on the trade-off between computational efficiency and accuracy. 

FHE enables privacy-preserving machine learning by allowing a server to process the client’s data in its encrypted form through an ML model. With FHE, the server learns neither the plaintext version of the input features nor the inference results. Only the client, using their secret key, can decrypt and access the results at the end of the service protocol.
FHE can also be applied to confidential blockchain services, ensuring that sensitive data in smart contracts remains encrypted and confidential while maintaining the transparency and integrity of the execution process.
Other applications of FHE include secure outsourcing of data analytics, encrypted database queries, privacy-preserving searches, efficient multi-party computation for digital signatures, and more.

This book is designed to help the reader understand how FHE works from the mathematical level. The book comprises the following four parts: 

$ $

\begin{itemize}
\item \textbf{\autoref{part:basic-math}:~\nameref{part:basic-math}} explains necessary background concepts for FHE, such as Group, Field, Order, Polynomial Ring, Cyclotomic Polynomial, Vectors and Matrices, Chinese Remainder Theorem, Taylor Series, Polynomial Interpolation, and Fast Fourier Transform.

\item \textbf{\autoref{part:pqc}:~\nameref{part:pqc}} explains well-known lattice-based cryptographic schemes, which are LWE, RLWE, GLWE, GLev, and GGSW cryptosystems.

\item \textbf{\textbf{\autoref{part:generic-fhe}:~\nameref{part:generic-fhe}}} explains the generic techniques of FHE adopted by many existing schemes, such as homomorphic addition, multiplication, modulus switching, and key switching. 


\item \textbf{\textbf{\autoref{part:fhe-schemes}:~\nameref{part:fhe-schemes}}} explains four widely used FHE schemes: TFHE, BFV, CKKS, and BGV, as well as their RNS-variant versions.
\end{itemize}

$ $

%These parts are designed in an incremental manner, and therefore understanding each part requires the understanding of its prior part(s). 

%$ $

This book is available both as an \href{https://arxiv.org/abs/2503.05136}{\textbf{arXiv PDF}} and on a \href{https://fhetextbook.github.io} {\textbf{tentative website}} (powered by \href{https://www.kodymirus.cz/overleaf-html-sample/main.html}{make4ht}). 
Please report any bugs or suggestions regarding the draft to the \href{https://github.com/fhetextbook/fhe-textbook/issues}{\textbf{Issues Board}}. As this book is an evolving open project, we welcome FHE experts to join us as collaborators and help expand the draft.



\subsubsection*{Acknowledgments}
Special thanks are extended to Robin Geelen (KU Leuven) for his thoughtful and dedicated comments, and to Yongwoo Lee (Inha University) for his general advice.

\thispagestyle{empty}

\newpage

\tableofcontents



%\newcommand{\subparagraph}{} % /usepackage{titlesec}
\titleformat*{\section}{\LARGE\bfseries\scshape}
\titleformat*{\subsection}{\Large\bfseries}
\titleformat*{\subsubsection}{\bfseries}
\titleformat*{\paragraph}{\itshape\subsubsectionfont}
\titleformat*{\subparagraph}{\large\bfseries}

% page header foot 
%\usepackage{fancyhdr}
%\pagestyle{fancy}
%\lhead{Security and Privacy in Cyber-Physical Systems: Foundations and Applications}
%\rfoot{Copyright \textcopyright 2016 by Wiley}
% \thispagestyle{fancy}, after \maketitle
\newcommand{\para}[1]{\vspace{0.05in}\noindent{\bf{#1}}}


\newcommand{\white}[1]{{\textcolor{white}{#1}}} % phantom upgrade

\newcommand{\gap}[1]{\text{ } \text{#1} \text{ }}

\newcommand{\tboxlabel}[1]{$\bm{\langle}$Summary~{#1}$\bm{\rangle}$}

\newcommand{\tboxdef}[1]{$\bm{\langle}$Definition~{#1}$\bm{\rangle}$}

\newcommand{\tboxtheorem}[1]{$\bm{\langle}$Theorem~{#1}$\bm{\rangle}$}

\newtheorem{proposition}{Proposition}



\newtcolorbox[blend into=tables]{mytable}[2][]{float=htb, halign=center,  title={#2}, every float=\centering, #1}


% $\hat Y = \frac{1}{1 + e^{-Z}}$.

% $Z = {w_1 \cdot X_1 + w_2 \cdot X_2 + \dots + w_n \cdot X_n + b}$



\clearpage

%\section{Background}

\section{Roots of Unity and Cyclotomic Polynomial over Integer Ring}
\label{sec:cyclotomic-polynomial-integer-ring}

In \autoref{sec:roots} and \autoref{sec:cyclotomic}, we learned about the definition and properties of the $\mu$-th roots of unity and the $\mu$-th cyclotomic polynomial over complex numbers (i.e., $X \in \mathbb{C}$) as follows: 

\begin{itemize}
\item \textbf{The $\bm \mu$-th roots of unity} are the solutions for $X^\mu = 1$ over $X \in \mathbb{C}$ (complex numbers). The formula for the $\mu$-th root of unity is $X = e^{2 \pi i k / \mu}$ for all integer $k$ such that $0 \leq k \leq \mu - 1$. 
\item \textbf{The primitive $\bm \mu$-th roots of unity (denoted as $\bm \omega$)} are those $\mu$-th roots of unity whose order (\autoref{subsec:order-def}) is $\mu$ (i.e., $\omega^{\mu} = 1$ and $\omega^{\frac{\mu}{2}} \neq 1$). 
\item Given any primitive $\mu$-th roots of unity $\omega$, it can generate all primitive $\mu$-th roots of unity by computing $\omega^{k'}$ such that $k'$ is an integer $0 < k' < \mu$ and $\textsf{gcd}(k', \mu) = 1$ (Theorem~\ref*{subsec:roots-theorem}.4 in \autoref{subsec:order-theorem}). 
\item \textbf{The $\bm \mu$-th cyclotomic polynomial} is defined as a polynomial whose roots are the primitive $\mu$-th roots of unity. That is, \[ \Phi_{\mu}(x) = \prod_{\omega \in P({\mu})} (x - \omega) = \prod_{\substack{0 \leq k \leq {\mu}-1,\\ \text{gcd}(k, {\mu}) = 1}} (x - \omega^k) \]
\end{itemize}

In this section, we will explain the $\mu$-th cyclotomic polynomial over $X \in \mathbb{Z}_p$ (integer ring), which is structured as follows: 

\begin{tcolorbox}[title={\textbf{\tboxdef{\ref*{sec:cyclotomic-polynomial-integer-ring}} Roots of Unity and Cyclotomic Polynomial over Integer Ring $\mathbb{Z}_p$}}]


\begin{itemize}
\item \textbf{The $\bm \mu$-th roots of unity (denoted as $\bm \omega$)} are the solutions for $X^\mu \equiv 1 \bmod p$. Note that these solutions are not $X = \omega^{2 \pi i k / \mu}$ (the formula for the solutions over $X \in \mathbb{C}$). 
\item \textbf{The primitive $\bm \mu$-th roots of unity} are defined as those $\mu$-th roots of unity whose order (\autoref{subsec:order-def}) is $\mu$ (i.e., $\omega^{\mu} \equiv 1 \bmod p$, and $\omega^{\lceil \frac{\mu}{2} \rfloor} \not\equiv 1 \bmod p$). 
\item Given any primitive $\mu$-th roots of unity $\omega$, it can generate all primitive $\mu$-th roots of unity by computing $\omega^{k'}$ such that $k'$ is an integer $0 < k' < \mu$ and $\textsf{gcd}(k', \mu) = 1$.
\item \textbf{The $\bm \mu$-th cyclotomic polynomial} is defined as a polynomial whose roots are the primitive $\mu$-th roots of unity. That is, \[ \Phi_{\mu}(x) = \prod_{\omega \in P({\mu})} (x - \omega) = \prod_{\substack{0 \leq k \leq {\mu}-1,\\ \text{gcd}(k, {\mu}) = 1}} (x - \omega^k) \]
\end{itemize}

\end{tcolorbox}

\begin{table}[h] %usepackage{array} 
\begin{tabular}{|c||c||c|}
\hline \hline
& \textbf{Polynomial over $\bm{X} \bm{\in} \bm{\mathbb{C}}$} & \textbf{Polynomial over $\bm{X} \in \bm{\mathbb{Z}}_{\bm{p}}$} \\ 
& \textbf{(Complex Number)} & \textbf{(Integer Ring)} \\ \hline \hline
\textbf{Definition}&All $X \in \mathbb{C}$ such that $X^\mu = 1$, (which are&All $X \in \mathbb{Z}_p$ such that $X^\mu \equiv 1 \bmod p$\\
\textbf{of the}&computed as $X = e^{2 \pi i k / \mu}$ for integer $k$&\\
\textbf{$\bm \mu$-th}&where $0 \leq k \leq \mu - 1$)&\\
\textbf{Root of Unity}&&\\\hline
\textbf{Definition}&Those $\mu$-th roots of unity $\omega$ such that&Those $\mu$-th roots of unity $\omega$ such that\\
\textbf{of the}&$\omega^{\mu} = 1$, and $\omega^{\frac{\mu}{2}} \neq 1$&$\omega^{\mu} \equiv 1 \bmod p$, and $\omega^{\frac{\mu}{2}} \not\equiv 1 \bmod p$\\
\textbf{Primitive}&&\\
\textbf{$\bm \mu$-th}&&\\
\textbf{Root of}&&\\
\textbf{Unity}&&\\\hline
\textbf{Definition}&\multicolumn{2}{|c|}{The polynomial whose roots are the $\mu$-th primitive roots of unity as follows:}\\
\textbf{of the}&\multicolumn{2}{|c|}{$ \Phi_{\mu}(x) = \prod_{\omega \in P(\mu)} (x - \omega) $  \text{ } (see Definition~\ref*{subsec:cyclotomic-def} in \autoref{subsec:cyclotomic-def})}\\
\textbf{$\bm \mu$-th}&\multicolumn{2}{|c|}{}\\
\textbf{Cyclotomic}&\multicolumn{2}{|c|}{}\\
\textbf{Polynomial}&\multicolumn{2}{|c|}{}\\\hline
\textbf{Finding}&For $\omega = e^{2 \pi i/ \mu}$, compute all $\omega^k$ such that&Find one satisfactory $\omega$ that is a root of\\
\textbf{Primitive}&$0 < k < \mu $ and $\textsf{gcd}(k, \mu) = 1$&the $\mu$-th cyclotomic polynomial, and\\
\textbf{$\bm \mu$-th}&(Theorem~\ref*{subsec:roots-theorem}.4 in \autoref{subsec:roots-theorem})&compute all $\omega^k \bmod p$ such that\\
\textbf{Roots of}&&$0 < k < \mu $ and $\textsf{gcd}(k, \mu) = 1$\\
\textbf{Unity}&&\\\hline\hline
\end{tabular}
\caption{The roots of unity and cyclotomic polynomials over $X \in \mathbb{C}$ v.s. over $X \in \mathbb{Z}_p$}
\label{tab:cyclotomic-polynomial-comparison}
\end{table}

Note that in the $\mu$-th cyclotomic polynomial in both cases of over $X \in \mathbb{C}$ and over $X \in \mathbb{Z}_p$, each of their roots $\omega$ (i.e., the primitive $\mu$-th root of unity) has the order $\mu$ (i.e., $\omega^{\mu} = 1$ over $X \in \mathbb{C}$, and $\omega^{\mu} \equiv 1 \bmod p$ over $X \in \mathbb{Z}_p$). Also note that each root $\omega$ can generate all roots of the $\mu$-th cyclotomic polynomial by computing $\omega^{k'}$ such that $\textsf{gcd}(k', \mu) = 1$.

\autoref{tab:cyclotomic-polynomial-comparison} compares the properties of the roots of unity and the $\mu$-th cyclotomic polynomial over $X \in \mathbb{C}$ (complex numbers) and over $X \in \mathbb{Z}_p$ (integer ring).






\subsection{Vandermonde Matrix with Roots of a Cyclotomic Polynomial}
\label{subsec:vandermonde-euler-integer-ring}

Theorem~\ref*{subsec:vandermonde-euler} (in \autoref{subsec:vandermonde-euler}) showed that $V \cdot V^T = n \cdot I^R_n$, where $V$ is the Vandermonde matrix $V = \mathit{Vander}(x_0, x_1, \cdots, x_{n-1})$, where each $x_i$ is the primitive $\mu$-th root of unity over $X \in \mathbb{C}$, where $\mu$ is a power of 2. In this subsection, we will show that the relation $V \cdot V^T = n \cdot I^R_n$ holds even if each $x_i$ is the primitive $\mu$-th root of unity over $X \in \mathbb{Z}_p$. In particular, we will prove Theorem~\ref*{subsec:vandermonde-euler}: 


\begin{tcolorbox}[title={\textbf{\tboxtheorem{\ref*{subsec:vandermonde-euler}} Vandermonde Matrix with Roots of  \text{(power-of-2)}-th Cyclotomic Polynomial}}]

Suppose we have an $n \times n$ (where $n$ is a power of 2) Vandermonde matrix comprised of $n$ distinct roots of the $\mu$-th cyclotomic polynomial over $X \in \mathbb{Z}_p$ (integer ring), where $\mu$ is a power of 2 and $n = \dfrac{\mu}{2}$. In other words, $V = \mathit{Vander}(x_0, x_1, \cdots, x_{n-1})$, where each $x_i$ is the root of $X^n + 1$ (i.e., the primitive $(\mu=2n)$-th roots of unity). Then, the following holds:

$V \cdot V^T = \begin{bmatrix}
0 & \cdots & 0 & 0 & n\\
0 & \cdots & 0 & n & 0\\
0 & \cdots & n & 0 & 0\\
\vdots & \iddots & \vdots & \vdots & \vdots \\
n & 0 & 0 & \cdots & 0\\
\end{bmatrix} = n \cdot I^R_n$

$ $

And $V^{-1} = n^{-1}\cdot V^T \cdot I_n^R$


\end{tcolorbox}
\begin{proof}
$ $
\begin{enumerate}
\item $V \cdot V^T$ is expanded as follows:

$V \cdot V^T = \begin{bmatrix}
1 & (\omega) & (\omega)^2 & \cdots & (\omega)^{n-1}\\
1 & (\omega^3) & (\omega^3)^2 & \cdots & (\omega^3)^{n-1}\\
1 & (\omega^5) & (\omega^5)^2 & \cdots & (\omega^5)^{n-1}\\
\vdots & \vdots & \vdots & \ddots & \vdots \\
1 & (\omega^{2n-1}) & (\omega^{2n-1})^2 & \cdots & (\omega^{2n-1})^{n-1}\\
\end{bmatrix} 
\cdot 
\begin{bmatrix}
1 & 1 & 1 & \cdots & 1\\
(\omega) & (\omega^3) & (\omega^5) & \cdots & (\omega^{2n-1})\\
(\omega)^2 & (\omega^3)^2 & (\omega^5)^2 & \cdots & (\omega^{2n-1})^2\\
\vdots & \vdots & \vdots & \ddots & \vdots \\
(\omega)^{n-1} & (\omega^3)^{n-1} & (\omega^5)^{n-1} & \cdots & (\omega^{2n-1})^{n-1}\\
\end{bmatrix} $

$ $


$=
\begin{bmatrix}
\sum\limits_{k=0}^{n-1} \omega^{2k} & \sum\limits_{k=0}^{n-1} \omega^{4k}  & \sum\limits_{k=0}^{n-1} \omega^{6k} & \cdots & \sum\limits_{k=0}^{n-1} \omega^{2nk} \\

\sum\limits_{k=0}^{n-1} \omega^{4k} & \sum\limits_{k=0}^{n-1} \omega^{6k}  & \sum\limits_{k=0}^{n-1} \omega^{8k} & \cdots & \sum\limits_{k=0}^{n-1} \omega^{2k(n+1)} \\

\sum\limits_{k=0}^{n-1} \omega^{6k} & \sum\limits_{k=0}^{n-1} \omega^{8k} & \sum\limits_{k=0}^{n-1} \omega^{10k} & \cdots & \sum\limits_{k=0}^{n-1} \omega^{2k(n+2)} \\

\vdots & \vdots & \vdots & \ddots & \vdots \\
\sum\limits_{k=0}^{n-1} \omega^{2nk} & \sum\limits_{k=0}^{n-1} \omega^{2(n+1)k} & \sum\limits_{k=0}^{n-1} \omega^{2(n+2)k} & \cdots & \sum\limits_{k=0}^{n-1} \omega^{2(n+n-1)k} \\

\end{bmatrix}$

$ $

, where $\omega$ (i.e., the primitive $(\mu=2n)$-th root of unity) has the order $2n$. 

$ $

\item Note that the $V \cdot V^T$ matrix's anti-diagonal elements are $\sum\limits_{k=0}^{n-1} \omega^{2nk}$. It can be seen that $\omega^{2n} \equiv 1 \bmod p$, because $\textsf{Ord}_p(\omega) = 2n$. Thus, the $V \cdot V^T$ matrix's every anti-diagonal element is $\sum\limits_{k=0}^{n-1} 1 = n$.

$ $

\item Next, we will prove that the $V \cdot V^T$ matrix has $0$ for all other positions than the anti-diagonal ones. In other words, we will prove the following: 

$\sum\limits_{k=0}^{n-1} \omega^{2k} = \sum\limits_{k=0}^{n-1} \omega^{4k} = \sum\limits_{k=0}^{n-1} \omega^{6k} = \gap{$\cdots$} = \sum\limits_{k=0}^{n-1} \omega^{2(n-1)k} = \sum\limits_{k=0}^{n-1} \omega^{2(n+1)k} = \gap{$\cdots$} = \sum\limits_{k=0}^{n-1} \omega^{2(2n-1)k} = 0$

$ $

The above is true, because in the particular case of the $(\mu=2n)$-th cyclotomic polynomial $X^n + 1$ (where $n$ is a power of 2), $\omega^{i} + \omega^{i+\frac{n}{2}} \equiv 0 \bmod p$ for each integer $i$ where $0 \leq i \leq n - 1$. Therefore, in the element $\sum\limits_{k=0}^{n-1} \omega^{2k}$, its one-half terms add with its other-half terms and their final summation becomes $0$. This is the same for all other following elements: 

$\sum\limits_{k=0}^{n-1} \omega^{4k}, \text{ } \sum\limits_{k=0}^{n-1} \omega^{6k}, \text{ } \gap{$\cdots$}, \text{ } \sum\limits_{k=0}^{n-1} \omega^{2(n-1)k}, \text{ } \sum\limits_{k=0}^{n-1} \omega^{2(n+1)k}, \text{ } \gap{$\cdots$}, \text{ } \sum\limits_{k=0}^{n-1} \omega^{2(2n-1)k}$

$ $

\item According to step 1 and 2, the $V \cdot V^T$ matrix has $n$ on its anti-diagonal positions and $0$ for all other positions.

\item Now we will derive the formula for $V^{-1}$. Given $V \cdot V^T = n \cdot I_n^R$, 

$V^{-1} \cdot V \cdot V^T = V^{-1} \cdot n \cdot I_n^R$

$V^T = V^{-1} \cdot n \cdot I_n^R$

$V^T \cdot I_n^R = V^{-1} \cdot n \cdot I_n^R  \cdot I_n^R$

$V^T \cdot I_n^R = V^{-1} \cdot n$ \textcolor{red}{\text{ } \# since $I_n^R  \cdot I_n^R = I_n$}

$ $

Now, there is one caveat: modulo operation does not support direct number division (as explained in \autoref{subsec:modulo-division}). This means that the formula $V^{-1} = \dfrac{V^T \cdot I_n^R}{n}$ in Theorem~\ref*{subsec:vandermonde-euler} (in \autoref{subsec:vandermonde-euler}) is inapplicable in our case, because our modulo $p$ arithmetic does not allow direct division of $V^T \cdot I_n^R$ by $n$. Therefore, we instead multiply $V^T \cdot I_n^R$ by the inverse of $n$ (i.e., $n^{-1}$). We continue as follows:

$V^T \cdot I_n^R = V^{-1} \cdot n$

$V^T \cdot I_n^R \cdot n^{-1}= V^{-1} \cdot n \cdot n^{-1}$

$V^{-1} = n^{-1}\cdot V^T \cdot I_n^R$

\end{enumerate}
\end{proof}

We finally proved that $V\cdot V^T = n \cdot I_n^R$, and $V^{-1} = n^{-1}\cdot V^T \cdot I_n^R$. Later in the BFV scheme (\autoref{sec:bfv}), we will use $V^{-1}$ to encode an integer vector into a vector of polynomial coefficients, and $V^T$ to decode it back to the integer vector (\autoref{subsec:bfv-batch-encoding}).

$ $

\para{Condition for $\bm \mu$:} Like in CKSS, it's worthwhile to note that the property $V\cdot V^T = n\cdot I_n^R$ does not hold if $\mu$ (denoting the $\mu$-th cyclotomic polynomial) is not a power of 2. In particular, step 3 of the proof does not hold anymore if $\mu$ is not a power of 2:

$\sum\limits_{k=0}^{n-1} \omega^{2k} \neq \sum\limits_{k=0}^{n-1} \omega^{4k} \neq \sum\limits_{k=0}^{n-1} \omega^{6k} \neq \gap{$\cdots$} \neq \sum\limits_{k=0}^{n-1} \omega^{2(n-1)k} \neq \sum\limits_{k=0}^{n-1} \omega^{2(n+1)k} \neq \gap{$\cdots$} \neq \sum\limits_{k=0}^{n-1} \omega^{2(2n-1)k} \neq 0$




%\clearpage
%\input{scratch-real}

\end{document}


%https://www.youtube.com/watch?v=vYKdh5oQ4Zw
