The GLWE cryptosystem is a generalized form to encompass both the LWE and RLWE cryptosystems. The GLWE cryptosystem's ciphertext is a tuple $(\{A_i\}_{i=0}^{k-1}, B)$, where $B = \sum\limits_{i=0}^{k-1}{(A_i \cdot S_i)} + \Delta \cdot M + E$. The public key $\{A_i\}_{i=0}^{k-1}$ and the secret key $\{S_i\}_{i=0}^{k-1}$ are a list of $k$ $(n-1)$-degree polynomials, each. The message $M$ and the noise $E$ are an $(n-1)$-degree polynomial, each. Like in LWE and GLWE, a new public key $A$ is created for each ciphertext, whereas the same secret key $S$ is used for all ciphertexts. In this section, we denote each ciphertext instance as $(\{A_i\}_{i=0}^{k-1}, B)$ instead of $(\{A_i\}_{i=0}^{k-1, \langle j \rangle}, B^{\langle j \rangle})$ for simplicity.




\subsection{Setup}
Let $t$ the size of plaintext, and $q$ the size of ciphertext, where $t < q$ ($t$ is much smaller than $q$) and $t | q$ (i.e., $t$ divides $q$). Randomly pick a list of $k$ $(n-1)$-degree polynomials as a secret key, where each polynomial coefficient is a randomly picked binary number in $\{0, 1\}$ (i.e., $\{S_i\}_{i=0}^{k-1} \xleftarrow{\$} \mathcal{R}_{\langle n, 2 \rangle}^k$). Let $\Delta = \left\lfloor\dfrac{q}{t}\right\rfloor$ the scaling factor of plaintext.

Notice that GLWE's setup parameters are similar to that of RLWE. One difference is that $S$ is not an $(n-1)$-degree polynomial encoding $n$ secret coefficients, but a list of $k$ such $(n-1)$-degree polynomials encoding total $n \cdot k$ secret coefficients.


\subsection{Encryption}
\label{subsec:glwe-enc}

Suppose we have an $(n-1)$-degree polynomial $M \in \mathcal{R}_{(n, t)}$ whose coefficients represent the plaintext numbers to encrypt.


\begin{enumerate}
\item Randomly pick a list of $k$ $(n-1)$-degree polynomials $\{A_i\}_{i=0}^{k-1} \xleftarrow{\$} \mathcal{R}_{\langle n,q \rangle}^{k}$ as a one-time public key.
\item Randomly pick a small polynomial $E \xleftarrow{\chi_\sigma} \mathcal{R}_{\langle n,q \rangle}$ as a one-time noise, whose $n$ coefficients are small numbers in $\mathbb{Z}_q$ randomly sampled from the Gaussian distribution $\chi_\sigma$. 
\item Scale $M$ by $\Delta$, which is to compute $\Delta \cdot M$. This converts $M \in \mathcal{R}_{\langle n, t \rangle}$ into $\Delta \cdot M \in \mathcal{R}_{\langle n,q \rangle}$.
\item Compute $B = \sum\limits_{i=0}^{k-1}{(A_i \cdot S_i)} + \Delta \cdot M + E \in \mathcal{R}_{\langle n,q \rangle}$. 
\end{enumerate}

$ $

The GLWE encryption formula is summarized as follows:

$ $

\begin{tcolorbox}[title={\textbf{\tboxlabel{\ref*{subsec:glwe-enc}} GLWE Encryption}}]

\textbf{\underline{Initial Setup}:} $\Delta = \left\lfloor\dfrac{q}{t}\right\rfloor$, $\{S_i\}_{i=0}^{k-1} \xleftarrow{\$} \mathcal{R}_{\langle n, 2 \rangle}^k$

$ $

$ $

\textbf{\underline{Encryption Input}:} $M \in \mathcal{R}_{\langle n, t \rangle}$, $\{A_i\}_{i=0}^{k-1} \xleftarrow{\$} \mathcal{R}_{\langle n,q \rangle}^{k}$, $E \xleftarrow{\chi_\sigma} \mathcal{R}_{\langle n,q \rangle}$

$ $

\begin{enumerate}

\item Scale up $M \longrightarrow \Delta M \text { } \in \mathcal{R}_{\langle n, q\rangle}$

\item Compute $B = \sum\limits_{i=0}^{k-1}{(A_i \cdot S_i)} + \Delta  M + E \text{ } \in \mathcal{R}_{\langle n,q \rangle}$

\item $\textsf{GLWE}_{S,\sigma}(\Delta M) = (\{A_i\}_{i=0}^{k-1}, B) \text{ } \in \mathcal{R}_{\langle n,q \rangle}^{k + 1}$ 



\end{enumerate}

\end{tcolorbox}

\subsection{Decryption}
\label{subsec:glwe-dec}

\begin{enumerate}
\item Given the ciphertext $(\{A_i\}_{i=0}^{k-1}, B)$ where $B = \sum\limits_{i=0}^{k-1}{(A_i \cdot S_i)} + \Delta \cdot M + E \in \mathcal{R}_{\langle n,q \rangle}$, compute $B - \sum\limits_{i=0}^{k-1}{(A_i \cdot S_i)} = \Delta \cdot M + E$. 
\item Round each coefficient of the polynomial $\Delta \cdot M + E \in \mathcal{R}_{\langle n,q \rangle}$ to the nearest multiple of $\Delta$ (i.e., round it as a base $\Delta$ number), which is denoted as $\lceil \Delta \cdot M + E \rfloor_{\Delta}$. This operation successfully eliminates $E$ and gives $\Delta \cdot M$. One caveat is that $E$'s each coefficient $e_i$ has to be $e_i < \dfrac{\Delta}{2}$ to be eliminated during the rounding. Otherwise, some of $e_i$'s higher bits will overlap the plaintext $m_i$ coefficient's lower bit and won't be eliminated during decryption, corrupting the plaintext $m_1$.
\item Compute $\dfrac{\Delta \cdot M} {\Delta}$, which is equivalent to right-shifting each polynomial coefficient in $\Delta \cdot M$ by $\text{log}_2 \Delta$ bits. 
\end{enumerate}

$ $

In summary, the GLWE decryption formula is summarized as follows:

$ $

\begin{tcolorbox}[title={\textbf{\tboxlabel{\ref*{subsec:glwe-dec}} GLWE Decryption}}]
\textbf{\underline{Decryption Input}:} $C = (\{A_i\}_{i=0}^{k-1}, B) \text{ } \in \mathcal{R}_{\langle n,q \rangle}^{k + 1}$

\begin{enumerate}
\item $\textsf{GLWE}^{-1}_{S,\sigma}(C) = B - \sum\limits_{i=0}^{k-1}{(A_i \cdot S_i)} = \Delta  M + E \text{ } \in \mathcal{R}_{\langle n,q \rangle}$

\item Scale down 
$\Bigl\lceil \dfrac{ \Delta  M + E }{\Delta}\Bigr\rfloor \bmod t = M \text{ } \in \mathcal{R}_{\langle n, t \rangle}$

\end{enumerate}

For correct decryption, every noise coefficient $e_i$ of polynomial $E$ should be: $e_i < \dfrac{\Delta}{2}$.

\end{tcolorbox}


\subsubsection{Discussion}
\begin{enumerate}
\item \textbf{LWE} is a special case of GLWE where the polynomial ring's degree $n = 1$. That is, all polynomials in $\{A_i\}_{i=0}^{k-1}, \{S_i\}_{i=0}^{k-1}, E$, and $M$ are zero-degree polynomial constants. Instead, there are $k$ such constants for $A_i$ and $S_i$, so each of them forms a vector.
\item \textbf{RLWE} is a special case of GLWE where $k = 1$. That is, the secret key $S$ is a single polynomial $S_0$, and each encryption is processed by only a single polynomial $A_0$ as a public key.
\item \textbf{Size of $\bm{n}$:} A large polynomial degree $n$ increases the number of the secret key's coefficient terms (i.e., $S_i = s_{i,0} + s_{i,1}X + \gap{$\cdots$} + s_{i, n-1}X^{n-1} $), which makes it more difficult to guess the complete secret key. The same applies to the noise polynomial $E$ and the public key polynomials $A_i$, thus making it harder to solve the search-hard problem (\autoref{subsec:lattice-overview}). Also, higher-degree polynomials can encode more plaintext terms in the same plaintext polynomial $M$, improving the throughput efficiency of processing ciphertexts.
%\item \textbf{Size of $\bm{q}$:} Increasing $q$ increases the ciphertext molulus space, making it hard to attack the cryptosystem. 
%Also, a large modulo space creates an enough separation between the plaintext bits and the noise bits in the ciphertext, which reduces the growth rate of the noise bits across ciphertext multiplications (will be discussed in \autoref{subsubsec:glwe-mult-plain-discussion}) and thus we can do a more number of ciphertext multiplications before the noise bits overflow to the plaintext bit area. 
\item \textbf{Size of $\bm{k}$:} A large $k$ increases the number of the secret key polynomials $(S_0, S_1, \gap{$\cdots$}, S_k)$ and the number of the one-time public key polynomials $(A_0, A_1, \gap{$\cdots$}, A_k)$, which makes it more difficult for the attacker to guess the complete secret keys. Meanwhile, there is only a single $M$ and $E$ polynomials per GLWE ciphertext, regardless of the size of $k$. 
\item \textbf{Reducing the Ciphertext Size:} The public key $\{A_i\}_{i=0}^{k-1}$ has to be created for each ciphertext, which is a big size. To reduce this size, each ciphertext can instead include the seed $d$ for the pseudo-random number generation hash function $H$. Then, the public key can be lively computed $k-1$ times upon each encryption \& decryption as $\{H(s), H(H(s)), H(H(H(s)), \gap{$\cdots$} \}$. Note that $H$, by nature, generates the same sequence of numbers given the same random initial seed $d$. 
\end{enumerate}

\subsection{An Alternative Version of GLWE}
\label{subsec:glwe-alternative}

The following is an alternative version of \tboxlabel{\ref*{subsec:glwe-enc}}, where the sign of each $A_iS_i$ is flipped in the encryption and decryption formula as follows: 

\begin{tcolorbox}[title={\textbf{\tboxlabel{\ref*{subsec:glwe-alternative}} An Alternative GLWE Cryptosystem}}]

\textbf{\underline{Initial Setup}:} $\Delta = \left\lfloor\dfrac{q}{t}\right\rfloor$, $\{S_i\}_{i=0}^{k-1} \xleftarrow{\$} \mathcal{R}_{\langle n, 2 \rangle}^k$

$ $

\par\noindent\rule{\textwidth}{0.4pt}


\textbf{\underline{Encryption Input}:} $M \in \mathcal{R}_{\langle n, t \rangle}$, $\{A_i\}_{i=0}^{k-1} \xleftarrow{\$} \mathcal{R}_{\langle n,q \rangle}^{k}$, $E \xleftarrow{\chi_\sigma} \mathcal{R}_{\langle n,q \rangle}$

$ $

\begin{enumerate}
\item Scale up $M \longrightarrow \Delta M \text { } \in \mathcal{R}_{\langle n, q\rangle}$

\item Compute $B = \textcolor{red}{-\sum\limits_{i=0}^{k-1}{(A_i \cdot S_i)}} + \Delta  M + E \text{ } \in \mathcal{R}_{\langle n,q \rangle}$

\item $\textsf{GLWE}_{S,\sigma}(\Delta M) = (\{A_i\}_{i=0}^{k-1}, B) \text{ } \in \mathcal{R}_{\langle n,q \rangle}^{k + 1}$ 

\end{enumerate}

\par\noindent\rule{\textwidth}{0.4pt}


\textbf{\underline{Decryption Input}:} $C = (\{A_i\}_{i=0}^{k-1}, B) \text{ } \in \mathcal{R}_{\langle n,q \rangle}^{k + 1}$

\begin{enumerate}
\item $\textsf{GLWE}^{-1}_{S,\sigma}(C) = B \text{ } \textcolor{red}{ + \sum\limits_{i=0}^{k-1}{(A_i \cdot S_i)}} = \Delta  M + E \text{ } \in \mathcal{R}_{\langle n,q \rangle}$

\item Scale down 
$\Bigl\lceil \dfrac{ \Delta  M + E }{\Delta}\Bigr\rfloor = M  \text{ } \in \mathcal{R}_{\langle n, t \rangle}$

\end{enumerate}

For correct decryption, every noise coefficient $e_i$ of polynomial $E$ should be: $e_i < \dfrac{\Delta}{2}$.

\end{tcolorbox}

Even if the $A_iS_i$ terms flip their signs, the decryption stage cancels out those terms by adding their equivalent double-sign-flipped terms; thus, the same correctness of decryption is preserved as in the original version.  





\subsection{Public Key Encryption}
\label{subsec:glwe-public-key-enc}


The encryption scheme in \autoref{subsec:glwe-enc} assumes that it is the secret key owner who encrypts each plaintext. In this section, we explain a public key encryption scheme where we create a public key counterpart of the secret key and anyone who knows the public key can encrypt the plaintext in such a way that only the secret key owner can decrypt it. The high-level idea is that a portion of the components to be used in the encryption stage is pre-computed at the setup stage and published as a public key. At the actual encryption stage, the public key is multiplied by an additional randomness ($U$) and added with additional noises ($E_1, \vec{E}_2$) to create unpredictable randomness to each encrypted ciphertext. The actual scheme is as follows: 

\begin{tcolorbox}[title={\textbf{\tboxlabel{\ref*{subsec:glwe-enc}} GLWE Public Key Encryption}}]

\textbf{\underline{Initial Setup}:} 
\begin{itemize}
\item The scaling factor $\Delta = \left\lfloor\dfrac{q}{t}\right\rfloor$
\item The secret key $\vec{S} = \{S_i\}_{i=0}^{k-1} \xleftarrow{\$} \mathcal{R}_{\langle n, 2 \rangle}^k$
\item The public key pair $(\mathit{PK}_1, \vv{\mathit{PK}}_2) \in \mathcal{R}_{\langle n, q \rangle}^{k+1}$ is generated as follows:

$\vec{A} = \{A_i\}_{i=0}^{k-1} \xleftarrow{\$} \mathcal{R}_{\langle n, q \rangle}^k$, \text{ } $E \xleftarrow{\sigma} \mathcal{R}_{\langle n, q \rangle}$

$\mathit{PK}_1 = \vec{A} \cdot \vec{S} + E \in \mathcal{R}_{\langle n, q \rangle}$

$\vv{\mathit{PK}}_2 = \vec{A} \in \mathcal{R}_{\langle n, q \rangle}^{k}$

\end{itemize}

\par\noindent\rule{\textwidth}{0.4pt}

\textbf{\underline{Encryption Input}:} $M \in \mathcal{R}_{\langle n, t \rangle}$, \text{ } $U \xleftarrow{\$} \mathcal{R}_{\langle n,2 \rangle}, \text{ } E_1 \xleftarrow{\sigma} \mathcal{R}_{\langle n,q \rangle}, \text{ } \vec{E}_2 \xleftarrow{\sigma} \mathcal{R}_{\langle n,q \rangle}^k$

$ $

\begin{enumerate}

\item Scale up $M \longrightarrow \Delta M \text { } \in \mathcal{R}_{\langle n, q\rangle}$

\item Compute the followings: 

$B = \mathit{PK}_1\cdot U + \Delta  M + E_1 \text{ } \in \mathcal{R}_{\langle n,q \rangle}$

$\vec{D} = \vv{{\mathit{PK}}}_2 \cdot U + \vec{E}_2 \in \mathcal{R}_{\langle n,q \rangle}^{k}$ \textcolor{red}{\text{ } \# $\vv{\mathit{PK}}_2 \cdot U$ multiplies $U$ to each element of $\vv{\mathit{PK}}_2$}

\item $\textsf{GLWE}_{S,\sigma}(\Delta M) = (\vec{D}, B) \text{ } \in \mathcal{R}_{\langle n,q \rangle}^{k+1}$ 



\end{enumerate}


\par\noindent\rule{\textwidth}{0.4pt}

\textbf{\underline{Decryption Input}:} $C = (\vec{D}, B) \text{ } \in \mathcal{R}_{\langle n,q \rangle}^{k + 1}$

\begin{enumerate}
\item $\textsf{GLWE}^{-1}_{S,\sigma}(C) = B - \vec{D} \cdot \vec{S} = \Delta  M + E_{\mathit{all}} \text{ } \in \mathcal{R}_{\langle n,q \rangle}$

\item Scale down 
$\Bigl\lceil \dfrac{ \Delta  M + E_{\mathit{all}} }{\Delta}\Bigr\rfloor = M  \text{ } \in \mathcal{R}_{\langle n, t \rangle}$

\end{enumerate}

For correct decryption, every noise coefficient $e_i$ of polynomial $E_{\mathit{all}}$ should be: $e_i < \dfrac{\Delta}{2}$.

\end{tcolorbox}


The equation in the 1st step of the decryption process is derived as follows:

$\textsf{GLWE}^{-1}_{S,\sigma}\bm{(}\text{ } C = (B, \vec{D}) \text{ }\bm{)} = B - \vec{D}\cdot\vec{S}$

$ = (\mathit{PK}_1\cdot U + \Delta  M + E_2) - (\vv{\mathit{PK}}_2 \cdot U + \vec{E}_2)\cdot \vec{S} $

$=  (\vec{A} \cdot \vec{S} + E)\cdot U + \Delta M + E_2 - (\vec{A}\cdot U)\cdot \vec{S} - \vec{E}_2\cdot\vec{S}$

$=  (U\cdot\vec{A}) \cdot \vec{S} + E \cdot U + \Delta M + E_2 - (U\cdot \vec{A}) \cdot \vec{S} - \vec{E}_2\cdot\vec{S}$

$= \Delta M + E \cdot U + E_2 - \vec{E}_2\cdot\vec{S}$

$= \Delta M + E_{\mathit{all}}$ \textcolor{red}{\text{ } \# where $E_{\mathit{all}} = E \cdot U + E_2 - \vec{E}_2\cdot\vec{S}$} 


$ $

\para{Security:} The GLWE encryption scheme's encryption formula (Summary~\ref{subsec:glwe-enc} in \autoref{subsec:glwe-enc}) was as follows: 

$\textsf{GLWE}_{S, \sigma}(\Delta M) = \bm{(} \text{ } \vec{A}, \text{ } B = \vec{A}\cdot\vec{S} + \Delta M + E \text{ }\bm{)}$

$ $

, where the hardness of the LWE and RLWE problems guarantees that guessing $\vec{S}$ is difficult given $\vec{A}$ and $E$ are randomly picked at each encryption. On the other hand, the public key encryption scheme is as follows: 

$\textsf{GLWE}_{S, \sigma}(\Delta M) = \bm{(} \text{ } \vec{D} = \vv{{\mathit{PK}}}_2 \cdot U + \vec{E}_2, \text{ } B = \mathit{PK}_1\cdot U + \Delta  M + E_1 \text{ } \bm{)}$

$ $

, where $\mathit{PK}_1$, $\vv{\mathit{PK}}_2$ are fixed and $U$, $E_1$, $\vec{E}_2$ are randomly picked at each encryption. Given the polynomial degree $n$ is large, both schemes provide the equivalent level of hardness to solve the problem. 